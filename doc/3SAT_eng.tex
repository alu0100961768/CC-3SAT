\documentclass{article}
\usepackage[utf8]{inputenc}
\usepackage{fixltx2e,fix-cm}
\usepackage{amssymb}
\usepackage{amsmath}
\usepackage{makeidx}
\usepackage{multicol}
\usepackage[english]{babel} 
\selectlanguage{english}




% Portada personalizada

\begin{document}
\begin{titlepage}
	\centering
	{\scshape\LARGE Complejidad Computacional\par}
	\vspace{1cm}
	{\scshape\Large El problema 3-Satisfiability\par}
	\vspace{1.5cm}
	{\huge\bfseries 3SAT\par}
	\vspace{2cm}
	{\Large\itshape Alberto Delgado Soler\par}
	{\Large\itshape Germán Alfonso Teixidó\par}
	{\Large\itshape Tomás González Martín\par}
	{\Large\itshape Pier Paolo Tarasco\par}
	  
	 \vfill

% Bottom of the page
	{\large \today\par}
\end{titlepage}


\section{Introduction}
\subsection{Description of the problem}
The 3-satisfiability it's a special case of the satisfiability (SAT), in which each clause has exactly 3 literals. The SAT problem consist of verifying if, given a boolean expression with variables and without quantifiers, there is an assignment of values to the variables such that the expression evaluates to true.


\subsection{SAT to 3-SAT}
Let $ U = \{u_1, u_2,...,u_n\}$ be a set of variables, and $ C = \{c_1, c_2, ...,, c_m\}$ a set of clauses, we would like to create a new set $C'$ of clauses of three literals in a new set $U'$, such that $C'$ it's satisfiability if and only if $C$ is satisfiability.
To construct $C'$ let's describe different cases, based on the number of variables in $U$ and the number of variables needed to complete the triplets.
Suppose that $C$ contains $k$ literals:
\begin{itemize}
    \item \bold{Case 1}. If $k=1$ then $C_i = z_i$. In this case we need two more variables $v_1$ and $v_2$ to obtain four new clauses of three literals each.
    $$ C' = \{v_1, v_2, z1\}, \{v_1, \overline {v_2}, z_1\}, \{\overline{v_1}, v_2, z1\}, \{\overline{v_1}, \overline{v_2}, z_1\} $$
    The reason of this is that $v_1$ as $v_2$ can be true or false. Because of this we have to represent each case (including the negated literals).
    \item \bold{Case 2}. If $k=2$ then $C_i = \{z_1, z_2\}$. We already have two literals, we only need one more, $v_1$. Thus, we add the two possible values of the new literal.
        $$ C' = \{v_1, z_1, z_2\}, \{\overline{v_1}, z_1, z_2\}$$
    \item \bold{Case 3}. If we have $k=3$ we don't need any extra literal. The SAT problem already is in 3SAT.
    \item \bold{Case 4}. If $k>3$ we have to create "$k-3$" additional variables, increasing the number of clauses to $k-2$. The set C' will be made of three parts. The first part of the set, formed by the first two $k$ literals, adding the first extra variable.
        $$ \{z_1, z_2, v_1\}$$
    The other triplets (except the last one) will be in the form
    $$ \{\overline{v_i}, z_{i+2}, v_{i+1}\}$$
    At last, the set is completed by:
        $$ \{\overline{v_{k-3}}, z_{k-1}, z_k\}$$
    Thus, for the fourth case, where $k>3$, we have the 3-SAT problem with this form:
        $$ C' = \{z_1, z_2, v_1\} \cup \{\overline{v_i}, z_{i+2}, v_{i+1}\} \cup \{\overline{v_{k-3}}, z_{k-1}, z_k\}$$

\end{itemize}

\subsection{Developed Code}
The code, developed in the programming language \textit{C++}.
\subsubsection{Structure of the program}
\begin{itemize}
    \item \textit{/include}: files \textit{*.h} which contains the declarations of the classes utilized in the program.
    \item \textit{/obj}: contains the executable of the program, besides the input and output files, united to the object files (\textit{*.o*}).
    \item \textit{/src}: files with the extension \textit{*.cpp} which contains the definition of the classes, and the main program.
\end{itemize}

\subsubsection{Description of the main files}
\begin{itemize}
    \item \textit{SAT}: Contains the formal definition of the SAT problem.
    \item \textit{SAT3}: Being a special case of the SAT, inherit from it, implementing the concatenations of the clauses described before.
    \item \textit{SAT2SAT3}: \textit{SAT to 3SAT} Converts the problem from SAT to 3SAT utilizing the capacity of concatenating \de \textit{SAT3}
    \item \textit{input}: file which presents the problem in the conjunctive normal form (CNF) following the standard, where:
    \begin {itemize}
        \item The first lines can have a C follow by a comment on each line.
        \item After the letter p followed by cnf and then the number of the number of variables and the number of clauses.
        \item Then there must be a line for each clause and, in each clause, the numbers of the variables that are in the clause are placed (with nothing or with a “-” in front to indicate that it is denied) and at the end of the clause the one "0" to indicate that the clause was terminated.

    \end{itemize}
\end{itemize}
\end{document}

